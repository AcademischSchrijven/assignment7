\documentclass[12pt, a4paper]{article}

\setlength\parskip{1em}
\setlength\parindent{0em}

\title{Assignment 7}

\author{Hendrik Werner s4549775}

\begin{document}
\maketitle

\clearpage
\section*{Feedback}
\begin{tabular}{l l}
	\textbf{Feedback for} & Mick Koomen\\
	\textbf{Exercise} & 6\\
	\textbf{Feedback from} & Hendrik Werner\\
\end{tabular}

\paragraph{Magazine and format}
It is a good idea to quote the journal itself as reference, when referring to the audience of that journal. I think the first part is really good but in the second part you mention the "APA style", which I do not now, without explanation. Through a bit of Internet research I found that this probably refers to the American Psychological Association's style. I still do not know what the style consists of though; and why you make the conclusion that you need to use this style.

You only mention that the journal also publishes research articles, but why does that necessitate the APA style? Maybe this is common knowledge and I just do not know this.

\paragraph{Message}
We were supposed to make this part up so I think you cannot really give much feedback here. I like the subject you chose.

\paragraph{Audience}
I agree with your assessment of the audience's assumed prior knowledge. You quoted the journal on this before so there is little room for interpretation.

Maybe it would be a good idea to refer back to the quote on this though. You could put it into a footnote and reference it here.

\paragraph{Relevance}
If you really invented this AI I am sure there would be a lot of people interested. For fundamental breakthroughs there is always an audience.
\end{document}
