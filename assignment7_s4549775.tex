\documentclass[12pt, a4paper]{article}

\usepackage[margin=5em]{geometry}

\setlength\parskip{1em}
\setlength\parindent{0em}

\title{Assignment 7}

\author{Hendrik Werner s4549775}

\begin{document}
\maketitle

\clearpage
\section*{Feedback}
\begin{tabular}{l l}
	\textbf{Feedback for} & Mick Koomen\\
	\textbf{Exercise} & 6\\
	\textbf{Feedback from} & Hendrik Werner\\
\end{tabular}

\paragraph{General}
\begin{itemize}
	\item There is a mistake in the second sentence of your paragraph on the structure ("is showed" must be "is shown").
	\item You use "A.I." everywhere in your text. I have not really seen this anywhere but researched this to be sure, and it seems that "AI" is customary. When using "A.I." most people are referring to the movie from 2001.
\end{itemize}

\paragraph{Magazine and format}
It is a good idea to quote the journal itself as reference, when referring to the audience of that journal. I think the first part is really good but in the second part you mention the "APA style", which I do not now, without explanation. Through a bit of Internet research I found that this probably refers to the American Psychological Association's style. I still do not know what the style consists of though; and why you make the conclusion that you need to use this style.

You only mention that the journal also publishes research articles, but why does that necessitate the APA style? Maybe this is common knowledge and I just do not know this.

\paragraph{Message}
We were supposed to make this part up so I think you cannot really give much feedback here. I like the subject you chose.

\paragraph{Audience}
I agree with your assessment of the audience's assumed prior knowledge. You quoted the journal on this before so there is little room for interpretation.

Maybe it would be a good idea to refer back to the quote on this though. You could put it into a footnote and reference it here.

\paragraph{Relevance}
If you really invented this AI I am sure there would be a lot of people interested. For fundamental breakthroughs there is always an audience.

\paragraph{Introduction}
I think your introduction is unnecessarily long. The whole part explaining the Turing Test and the derivation thereof should probably be moved into another section.
This would also solve the problem of the weird structure of you introduction. You talk about the article and your AI in the beginning and end, but in the middle there is this big part about the details of the Turing Test.

\paragraph{Structure}
As I said I would personally not put the discussion of the specifics of the Turing Test in the introduction, especially if you want to start with the history of AI after that. The reasoning given for the structure does not cover why you chose this layout. You want to "show the reader what is means if an A.I. would pass it". You do not need to explain the methodology in detail for that, especially not in the introduction.

As you acknowledge yourself the audience cannot be assumed to have AI specific background knowledge. You also mention that you put the "more technical part" later in the article to first provide an introduction to the reader. I would definitely put that under "technical parts".

Other than that the proposed reasoning and structure are sound.
\end{document}
